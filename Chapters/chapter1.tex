%!TEX root = ../template.tex
%%%%%%%%%%%%%%%%%%%%%%%%%%%%%%%%%%%%%%%%%%%%%%%%%%%%%%%%%%%%%%%%%%%
%% chapter1.tex
%% NOVA thesis document file
%%
%% Chapter with introduction
%%%%%%%%%%%%%%%%%%%%%%%%%%%%%%%%%%%%%%%%%%%%%%%%%%%%%%%%%%%%%%%%%%%

\typeout{NT FILE chapter1.tex}%

\chapter{Introduction}
\label{cha:introduction}

This chapter includes a concise overview of the topic of this dissertation. 
Section \ref{sec:motivation} introduces the motivation behind this thesis, while Section \ref{sec:context} describes its context.
Following, Section \ref{sec:problem_description} presents the problem and the main objectives accomplished with this study. 
The contributions of this dissertation are summarized in Section \ref{sec:contributions}.
Finally, in Section \ref{sec:document_structure}, the report organization is outlined.
%complementing with  

\section{Motivation}
\label{sec:motivation}
This project studies the exploration of luxury glass from ancient Rome.
The central motivation of this dissertation is the research of remains and artefacts, not
only to preserve the Cultural Heritage (\gls{CH}) but also to convey ancestral heritage knowledge to everyone, including future generations. 
The growing integration of digital tools with arts and material culture enhances online 
facilities for users and democratizes its access. Additionally, it cultivates a deeper understanding of ancient cultures and traditions, allowing more people to engage with
\gls{CH} in meaningful ways. Moreover, by integrating technology with these archaeological 
discoveries, the goal is to raise the value of this archaeological site and make it accessible in an immersive and engaging \gls{VR} experience.


\section{Context}
\label{sec:context}

This thesis was developed under the umbrella of an interdisciplinary collaboration among three institutions.
The NOVA LINCS\footnote{\url{https://nova-lincs.di.fct.unl.pt/}}, \gls{VICARTE}\footnote{\url{https://vicarte.org/about-us/}}, and archaeologists from the Roman Ruins of Troia.
\gls{VICARTE} is a research unit focused on glass and ceramics, established as a partnership between \gls{NOVA} and the Faculty of Fine Arts, University of Lisbon, with contributions from the Faculty of Fine Arts, University of Oporto, School of Arts and Design, Polytechnic Institute of Leiria, and Cultural Institutions.
The research carried out at \gls{VICARTE} is structured on two interconnecting pillars: the study of Glass and Ceramics in Contemporaneity and in \gls{CH}. 
Historically, the Roman Ruins of Troia are recognized as the largest fish-salting production complex from the Roman period, active from the beginning of the 1st century until the 6th century.
The site of Troia is located on the southwestern coast of Portugal, on a sand embankment between the estuary of the Sado River and the Atlantic Ocean~\cite{pinto2018reassessment}.

The archaeological site stretches for about two kilometers along the Sado River estuary, and contains thirty fish-salting workshops~\cite{hocquet2015fish}.
With its significant historical and cultural value, the site of Troia has strongly influenced the economic and cultural development of the surrounding region up to the present.
In 2005, a protocol was established that enabled the hiring of an archaeological team responsible for the preservation, maintenance, and enhancement of the Troia archaeological remains~\cite{pinto2014ruinas}. 
Two years later, a project was initiated to make the Roman ruins accessible to visitors, including the creation of pathways with interpretative panels. The Roman Ruins of Troia were opened to the public in 2011.
Currently, artefact collections from Troia are exhibited in various institutions, including the \gls{NMA}\footnote{\url{https://www.museunacionalarqueologia.gov.pt/}}, the Museum of Archaeology and Ethnography of the District of Setúbal\footnote{\url{https://maeds.amrs.pt/}}, and the City Council of Grândola\footnote{\url{https://www.cm-grandola.pt/}}. The specific glass artefacts under study are housed at the \gls{NMA}.


\section{Problem Description and Objectives}
\label{sec:problem_description} 
This thesis concentrates on the study of glass relics discovered in the tomb of a wealthy woman who lived in the Troia Peninsula. The data collected from the studies and excavations of this site include 
data from the artefacts, excavation, and intervention process, such as images of the discovered glasses both before and after conservation. 
The preservation intervention work was carried out at \gls{VICARTE}, that provided access to the images mentioned.

This project comprises two digital tools developed in parallel, complemented by a third component responsible for managing communication between them.

Primarily, a data repository was designed to compile and organize findings from a Troia funerary enclosure, focusing specifically on glass objects found inside. A database was structured to store this data. 
This database stores essential objects' information, such as object ID, dimensions, location details, object utility, shape, 
conservation status, and provenance. It serves as a data source for the \gls{VE}.

The second tool, and the main focus of this dissertation, is a \gls{VR} experience focusing on an immersive visit to an existing physical 
artefact, to enable visitors to handle precious fragile antiquities, added to the environment as \gls{3D} models. 
Two of these object models are already completed and were supplied by the researchers involved in the project.
Furthermore, the application provides users with the option to experience a simulation of the original appearance of the glass artefacts.
In this environment, an interactive map of the Troia occidental site was integrated, featuring tooltips, points of interest, and the \gls{3D} model of the tomb within the funerary enclosure model positioned above the map, and developed from a photogrammetry capture. 
The map is the base ground plane on which users can walk and explore, with an additional functionality that enables toggling to another layer of the site.
The user interaction with the environment is supported through the use of the \gls{VR} headset display.

In addition, a backend implementation serves as the backbone of the two interactive tools described, supplying the \gls{VE} with informative data retrieved from the repository.

\section{Main Contributions}
\label{sec:contributions}

The main contributions achieved in this dissertation are:

\begin{itemize}
   \item  \textbf{Data Repository} - A comprehensive database containing all available data, accessible to users through interactions with the \gls{VR} environment.
   \item \textbf{\gls{VR} Immersive Experience} - Enables users to view and explore the funerary enclosure model with the map as ground plane, and interact with and manipulate virtual \gls{3D} objects within.  An innovative feature is the opportunity for users to virtually interact with glass-made historical relics, offering a sense of traveling back in time. Users can also navigate through a user-friendly map that enhances the feeling of presence within this authentic \gls{CH} space. 
   Additionally, the system delivers an engaging and interactive experience that enriches their understanding and interest in Roman heritage, making learning both enjoyable and meaningful.
   \item \textbf{Research Paper Publication and Presentation} - During the dissertation, a short paper based on the developed work was written, submitted, and accepted for publication at the \textit{SUMAC Workshop on Analysis, Understanding, and Promotion of Heritage Contents}\footnote{\url{https://sumac-workshops.github.io/2025/}}, part of the ACM Multimedia 2025 Conference\footnote{\url{https://acmmm2025.org/}}, to be held in Dublin, Irland. The corresponding workshop and poster presentation will take place on 27 October 2025.
\end{itemize}


\section{Document Structure}
\label{sec:document_structure}

This document is structured into seven chapters:

\begin{enumerate}
\item \textbf{Introduction:} The first chapter introduces the theme of the dissertation and explains its main focus.

\item \textbf{Fundamental Concepts:} This chapter provides background on the key areas of this thesis and explains significant concepts that were explored throughout the work.

\item \textbf{Related Work:} This segment consists of the research of similar projects and analysis of digital tools from relevant studies that can complement this dissertation.

\item \textbf{System Design:} This chapter presents the requirements and design decisions that guided the system's implementation.

\item \textbf{Implementation Details:} This chapter describes the implementation process. Code listings are included to illustrate technical details, while figures demonstrate the system’s functionalities.

\item \textbf{User Evaluation:} The following chapter describes the evaluation methodology, including the questionnaires and procedures used to assess user experience in the \gls{VE} developed. The chapter also presents and analyses the collected results, supported by graphics.

\item \textbf{Conclusions:} The final chapter summarizes the achieved goals,  encountered challenges, and possible future improvements. 
\end{enumerate}