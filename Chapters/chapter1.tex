%!TEX root = ../template.tex
%%%%%%%%%%%%%%%%%%%%%%%%%%%%%%%%%%%%%%%%%%%%%%%%%%%%%%%%%%%%%%%%%%%
%% chapter1.tex
%% NOVA thesis document file
%%
%% Chapter with introduction
%%%%%%%%%%%%%%%%%%%%%%%%%%%%%%%%%%%%%%%%%%%%%%%%%%%%%%%%%%%%%%%%%%%

\typeout{NT FILE chapter1.tex}%

\chapter{Introduction}
\label{cha:introduction}

This chapter includes a concise overview of the topic of this dissertation. 
Section \ref{sec:motivation} introduces the motivation behind this thesis, while Section \ref{sec:context} describes its context.
Following, Section \ref{sec:problem_description_and_objectives} presents the problems to solve and the main objectives of this study. 
The expected contributions of this dissertation are discussed in Section \ref{sec:expected_contributions}.
Finally, in Section \ref{sec:document_structure}, the organization of the report is outlined.
%complementing with  

\section{Motivation}
\label{sec:motivation}
Archaeology offers methods for interpreting human behavior and societal structures. By studying the past, archaeology provides insights that can be applied to the present~\footnote{\url{https://www.arch.ox.ac.uk/why-archaeology-matters}}.
Furthermore, it allows us to analyse the impacts of a changing world and the interactions among diverse countries, regions, and cultures.

This project will study the exploration of luxury glass from ancient Rome.
The central motivation of this dissertation is the research of remains and artefacts, not
only to preserve the \gls{CH} but also to transmit ancestral patrimony knowledge to everyone, including future generations. 
The growing integration of digital tools with arts enhances online 
facilities for users and democratizes access to art. Additionally, it cultivates a deeper understanding of ancient cultures and traditions, allowing more people to engage with
\gls{CH} in meaningful ways. Moreover, by integrating technology with these archaeological 
discoveries we want to raise the value of this archeology site and make it accessible on a global scale.


\section{Context}
\label{sec:context}

This thesis will be developed under the umbrella of an interdisciplinary collaboration between NOVA LINCS and \gls{VICARTE}\footnote{\url{https://vicarte.org/about-us/}}.
\gls{VICARTE} is a glass and ceramic research unit created as a partnership between the NOVA School of Science and Technology and the Faculty of Fine Arts, University of Lisbon, with members from the Faculty of Fine Arts, University of Oporto, the School of Arts and Design, Polytechnic Institute of Leiria, and Cultural Institutions.
The research at \gls{VICARTE} is based on two interconnecting pillars: Glass and Ceramics in Contemporaneity and in \gls{CH}. 

Historically, the Roman ruins of Troia represent the largest known fish-salting production complex, built in the first half of the first century, and continuously occupied until the 6th century. 
The site of Troia is located on the southwestern coast of Portugal, on a sand embankment between the estuary of the Sado River and the atlantic ocean~\cite{pinto2018reassessment}.

Today, this location is owned by Sonae Capital Group\footnote{\url{https://www.sonae.pt/pt/}} and is associated with the Troia Resort\footnote{\url{https://www.troiaresort.pt/en/troia-roman-ruins/}}.
The ruins extend along two kilometers of the Sado River estuary, comprising twenty-five fish-salting workshops~\cite{hocquet2015fish}.
A site of outstanding  universal value, with a unique magnitude, which has influenced the economy of an entire region and its development up to the present day~\cite{unesco_troia}.
In 2005, a protocol was established that enabled the hiring of an archaeological team responsible for the preservation, maintenance, and enhancement of the Troia archaeological remains~\cite{pinto2014ruinas}. In 2007,
a new project was launched to provide public access to the Roman ruins, leading to the installation of visitor pathways with interpretation panels. The site officially opened to the public in February 2011. 
Currently, artefacts collections from Troia are exhibited in various institutions, including the National Museum of Archaeology\footnote{\url{https://www.museunacionalarqueologia.gov.pt/}}, the Museum of Archaeology and Ethnography of the District of Setúbal\footnote{\url{https://maeds.amrs.pt/}}, and the City Council of Grândola\footnote{\url{https://www.cm-grandola.pt/}}. The specific glass artefacts under study are housed at the National Museum of Archaeology.



%colocar em footnote ou introduzir antes VICARTE(Glass and Ceramic for the Arts)
%bibliografia ? 
%legenda sonae?
%https://www.portugalresident.com/roman-troia-discovering-a-large-fish-salting-production-centre/    
%FOTOS PUBLICAS ESCAVACOES

\section{Problem Description and Objectives}
\label{sec:problem_description_and_objectives} 
%, this concept will be explained better in section X.
This study will concentrate on glass relics discovered in the mausoleum of a wealthy woman who lived in the Troia Peninsula. The data collected from the excavations include 
images of the discovered glasses both before and after conservation, and photographs documenting the entire excavation process. 
This information was provided by \gls{VICARTE}.

This project aims to develop three digital tools in parallel, with a subsequent phase dedicated to 
integrating them into a cohesive web app.


Primarily, a data repository that compiles and organizes and findings from 
the Troia excavations, focusing specifically on glass objects. A database will be needed to store this data. 
This database will store essential information such as an object ID, location points, object utility,
conservation status, origin, symbolic and decorative meanings, and comparisons with similar glass artefacts, using characteristics such as shape and uniqueness. 
This repository will serve as a data source for the other tools.


The second tool will be an interactive map of the Troia Peninsula, involving several layers of information, positioning each 
artefact found inside a grave. These layers will contain not only data from excavations but also, 
studied evidence enumerated above in the first development phase, the data repository.
 
Furthermore, an \gls{VR} experience focusing on immersing a visit of an existing physical 
artefact to enable visitors to handle precious fragile antiquities, added to the environment as \gls{3D} models. Some of these models are already completed and will be supplied by the researchers involved in the project.
It should use emerging display methods, such as the ones provided by \gls{VR} glasses.
The database will be the backbone of the two interactive tools described, feeding the interactive map with data and supporting the \gls{VR} development.

\section{Expected Contributions}
\label{sec:expected_contributions}


By the end of the development of this dissertation, the following contributions are expected:


\begin{itemize}
   \item \textbf{Virtual Platform:}
   \begin{itemize}
      \item \textbf{Interactive Map} - User-friendly map that allows users to explore geographical and historical locations interactively.
      \item \textbf{3D Object Manipulation} - Enables users to interact with and manipulate virtual 3D objects, utilizing the implementation of \hyperref[sec:marcia_thesis]{Márcia Campanha's thesis}~\cite{campanha2024heritage}.
      \item  \textbf{Data Repository} - A comprehensive database containing all available data, accessible to users through interactions with the map or the \gls{3D} models.
   \end{itemize}
   \item \textbf{User's Cultural Enrichment} - Provide users an engaging and interactive experience that enriches their understanding and interest
   in Roman \gls{CH}, making learning both enjoyable and meaningful.
   On top of that, through the integration of \gls{VR}, users will have the opportunity to virtually interact with glass made historical objects.
\end{itemize}


\section{Document Structure}
\label{sec:document_structure}


This document is divided into five chapters:

\begin{enumerate}
  \item \textbf{Introduction:} The first chapter introduces the theme of the dissertation and explains its main focus.
  
  \item \textbf{Fundamental Concepts:} This chapter provides background on the key areas of this thesis and explains important concepts that will be explored throughout the work.
  
  \item \textbf{Related Work:} This segment consists on the research of similar projects and analysis of digital tools from relevant studies that can complement this dissertation.

  \item \textbf{Proposed Work:} This chapter introduces a specific approach to achieve the project's goals and outlines how this will be implemented in practice.
\end{enumerate}
