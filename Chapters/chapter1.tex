%!TEX root = ../template.tex
%%%%%%%%%%%%%%%%%%%%%%%%%%%%%%%%%%%%%%%%%%%%%%%%%%%%%%%%%%%%%%%%%%%
%% chapter1.tex
%% NOVA thesis document file
%%
%% Chapter with introduction
%%%%%%%%%%%%%%%%%%%%%%%%%%%%%%%%%%%%%%%%%%%%%%%%%%%%%%%%%%%%%%%%%%%

\typeout{NT FILE chapter1.tex}%

\chapter{Introduction}
\label{cha:introduction}

This chapter includes a concise overview of the topic of this dissertation. 
Section \ref{sec:motivation} introduces the motivation behind this thesis, while Section \ref{sec:context} describes its context.
Following, Section \ref{sec:problem_description} presents the problem and the main objectives accomplished of this study. 
The contributions of this dissertation are summarized in Section \ref{sec:contributions}.
Finally, in Section \ref{sec:document_structure}, the report organization is outlined.
%complementing with  

\section{Motivation}
\label{sec:motivation}
Archaeology offers methods for interpreting human behavior and societal structures. By studying the past, archaeology provides insights that can be applied to the present~\footnote{\url{https://www.arch.ox.ac.uk/why-archaeology-matters}}.
Furthermore, it allows us to analyse the impacts of a changing world and the interactions among diverse countries, regions, and cultures informing the present.

This project studies the exploration of luxury glass from ancient Rome.
The central motivation of this dissertation is the research of remains and artefacts, not
only to preserve the \gls{CH} but also to transmit ancestral patrimony knowledge to everyone, including future generations. 
The growing integration of digital tools with arts and material culture enhances online 
facilities for users and democratizes its access. Additionally, it cultivates a deeper understanding of ancient cultures and traditions, allowing more people to engage with
\gls{CH} in meaningful ways. Moreover, by integrating technology with these archaeological 
discoveries, the goal is to raise the value of this archaeological site and make it accessible for an immersive and engaging \gls{VR} experience.


\section{Context}
\label{sec:context}

This thesis will be developed under the umbrella of an interdisciplinary collaboration between NOVA LINCS\footnote{\url{https://nova-lincs.di.fct.unl.pt/}} and \gls{VICARTE}\footnote{\url{https://vicarte.org/about-us/}}.
\gls{VICARTE} is a glass and ceramic research unit created as a partnership between the \gls{NOVA} and the Faculty of Fine Arts, University of Lisbon, with members from the Faculty of Fine Arts, University of Oporto, the School of Arts and Design, Polytechnic Institute of Leiria, and Cultural Institutions.
The research at \gls{VICARTE} is based on two interconnecting pillars: Glass and Ceramics in Contemporaneity and in \gls{CH}. 

Historically, the Roman ruins of Troia represent the largest known fish-salting production complex, built in the first half of the first century, and continuously occupied until the 6th century. 
The site of Troia is located on the southwestern coast of Portugal, on a sand embankment between the estuary of the Sado River and the Atlantic Ocean~\cite{pinto2018reassessment}.

The ruins extend along two kilometers of the Sado River estuary, comprising twenty-five fish-salting workshops~\cite{hocquet2015fish}.
A site of outstanding value, with a unique magnitude, which has influenced the economy of an entire region and its development up to the present day~\cite{unesco_troia}.
In 2005, a protocol was established that enabled the hiring of an archaeological team responsible for the preservation, maintenance, and enhancement of the Troia archaeological remains~\cite{pinto2014ruinas}. In 2007,
a new project was launched to provide public access to the Roman ruins, leading to the installation of visitor pathways with interpretation panels. The site officially opened to the public in February 2011. 
Currently, artefact collections from Troia are exhibited in various institutions, including the National Museum of Archaeology\footnote{\url{https://www.museunacionalarqueologia.gov.pt/}}, the Museum of Archaeology and Ethnography of the District of Setúbal\footnote{\url{https://maeds.amrs.pt/}}, and the City Council of Grândola\footnote{\url{https://www.cm-grandola.pt/}}. The specific glass artefacts under study are housed at the National Museum of Archaeology.



%colocar em footnote ou introduzir antes VICARTE(Glass and Ceramic for the Arts)
%bibliografia ? 
%legenda sonae?
%https://www.portugalresident.com/roman-troia-discovering-a-large-fish-salting-production-centre/    
%FOTOS PUBLICAS ESCAVACOES

\section{Problem Description and Objectives}
\label{sec:problem_description} 
%, this concept will be explained better in section X.
This study concentrate on glass relics discovered in the tomb of a wealthy woman who lived in the Troia Peninsula. The data collected from the studies and excavations include 
data from the artifacts, excavation, and intervention process, such as images of the discovered glasses both before and after conservation. 
The preservation intervention work was carried out at \gls{VICARTE}, that provided access to the images mentioned.

This project comprises two digital tools developed in parallel, complemented by a third component responsible for managing communication between them.

Primarily, a data repository that compiles and organizes findings from 
a Troia funerary enclosure, focusing specifically on glass objects found inside was designed. A database will be needed to store this data. 
This database will store essential objects' information, such as object ID, location details, object utility,
conservation status, origin, symbolic and decorative meanings, and comparisons with similar glass artefacts, using characteristics such as shape and uniqueness, defined with a link to the parallel discovery. 
This repository will serve as a data source for the \gls{VE}.

The second tool, and the main focus of this dissertation, is a \gls{VR} experience focusing on an immersive visit to an existing physical 
artefact to enable visitors to handle precious fragile antiquities, added to the environment as \gls{3D} models. 
Two of these object models are already completed and were supplied by the researchers involved in the project.
Furthermore, the application provides users with the option to experience a simulation of the original appearance of the glass artefacts.
In this environment, an interactive map of the Troia Occidental site is integrated, featuring tooltips, points of interest, and the grave within the funerary enclosure model positioned above the map. 
The map is the base ground plane on which users can walk and explore, with an additional functionality that enables toggling to another layer of the site.
The user interaction with the environment is supported through the use of the \gls{VR} headset display.

In addition, a backend implementation serves as the backbone of the two interactive tools described, supplying the \gls{VE} with informative data retrieved from the repository.

\section{Main Contributions}
\label{sec:contributions}

The main contributions achieved in this dissertation are:

\begin{itemize}
   \item  \textbf{Data Repository} - A comprehensive database containing all available data, accessible to users through interactions with the \gls{VR} environment.
   \item \textbf{\gls{VR} Immersive Experience} - Enables users to view and explore the funerary enclosure model with the map as ground plane, and interact with and manipulate virtual \gls{3D} objects within.  An innovative feature is the opportunity for users to virtually interact with glass-made historical relics, offering a sense of traveling back in time. Users can also navigate through a user-friendly map that enhances the feeling of presence within this authentic \gls{CH} space. 
   Additionally, the system delivers an engaging and interactive experience that enriches their understanding and interest in Roman heritage, making learning both enjoyable and meaningful.
   \item \textbf{Publication of a Paper} - During the dissertation, a short paper based on the developed work was written, submitted, and accepted for publication at the \textit{SUMAC Workshop on Analysis, Understanding, and Promotion of Heritage Contents}\footnote{\url{https://sumac-workshops.github.io/2025/}}. The corresponding workshop and poster presentation will take place on 27 October 2025 in Dublin.
\end{itemize}


\section{Document Structure}
\label{sec:document_structure}

This document is structured into six chapters:

\begin{enumerate}
\item \textbf{Introduction:} The first chapter introduces the theme of the dissertation and explains its main focus.

\item \textbf{Fundamental Concepts:} This chapter provides background on the key areas of this thesis and explains significant concepts that were explored throughout the work.

\item \textbf{Related Work:} This segment consists of the research of similar projects and analysis of digital tools from relevant studies that can complement this dissertation.

\item \textbf{System Design and Implementation:} This chapter describes the architecture, design choices, and the implementation process. Code listings are included to illustrate technical details, while figures demonstrate the system’s functionalities.

\item \textbf{User Evaluation:} The following chapter describes the evaluation methodology, including the questionnaires and procedures used to assess user experience in the \gls{VE} developed. The chapter also presents and analyzes the collected results, supported by graphics.

\item \textbf{Conclusions:} The final chapter summarizes the achieved goals,  encountered challenges, and possible future improvements. 
\end{enumerate}