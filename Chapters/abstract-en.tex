%!TEX root = ../template.tex
%%%%%%%%%%%%%%%%%%%%%%%%%%%%%%%%%%%%%%%%%%%%%%%%%%%%%%%%%%%%%%%%%%%%
%% abstract-en.tex
%% NOVA thesis document file
%%
%% Abstract in English([^%]*)
%%%%%%%%%%%%%%%%%%%%%%%%%%%%%%%%%%%%%%%%%%%%%%%%%%%%%%%%%%%%%%%%%%%%

\typeout{NT FILE abstract-en.tex}%

%Regardless of the language in which the dissertation is written, usually there are at least two abstracts: one abstract in the same language as the main text, and another abstract in some other language.

%The abstracts' order varies with the school.  If your school has specific regulations concerning the abstracts' order, the \gls{novathesis} (\LaTeX) template will respect them.  Otherwise, the default rule in the \gls{novathesis} template is to have in first place the abstract in \emph{the same language as main text}, and then the abstract in \emph{the other language}. For example, if the dissertation is written in Portuguese, the abstracts' order will be first Portuguese and then English, followed by the main text in Portuguese. If the dissertation is written in English, the abstracts' order will be first English and then Portuguese, followed by the main text in English.
%
%However, this order can be customized by adding one of the following to the file \verb+5_packages.tex+.

%\begin{verbatim}
  %  \ntsetup{abstractorder={<LANG_1>,...,<LANG_N>}}
 %   \ntsetup{abstractorder={<MAIN_LANG>={<LANG_1>,...,<LANG_N>}}}
%\end{verbatim}

%For example, for a main document written in German with abstracts written in German, English and Italian (by this order) use:
%\begin{verbatim}
 %   \ntsetup{abstractorder={de={de,en,it}}}
%\end{verbatim}

%Concerning its contents, the abstracts should not exceed one page and may answer the following questions (it is essential to adapt to the usual practices of your scientific area):

Nowadays, the significant historical and cultural heritage of Tróia remains largely unknown. 
This project studies luxury glass from ancient Rome, found in a luxurious grave.


The goal of this dissertation is to raise awareness of this valuable discovery through the development of several digital tools 
that will support the dissemination of this study, contribute to preserving its cultural value, and share its findings. 
This challenging project is currently starting and aims for future continuity in its development, with a great potential for further progress. 
This theme touches on various areas of Computer Science and Engineering, some of which are not covered in classes, making it a great opportunity to expand my technical knowledge. 
Additionally, it will help improve soft skills such as autonomy, creativity, and critical thinking by planning and structuring the project's different steps and integrating its components.


 The proposed solution involves the development of some interacting platforms, including an interactive map of the Tróia Peninsula, a repository featuring literature on Tróia's excavations, 
 and an augmented reality (AR) experience focusing on "augmenting" a visit to an existing physical object. 
 In the end, all these elements will be integrated into a digital environment as 3D models. 
 These technologies will enrich the experience of online visitors of the Tróia ruins, allowing them to interact with rare luxury artifacts from the Roman Empire. 
 Furthermore, positively impact people and increase their knowledge about this important archeological site while preserving its cultural legacy.

%\end{enumerate}

% Palavras-chave do resumo em Inglês
% \begin{keywords}
% Keyword 1, Keyword 2, Keyword 3, Keyword 4, Keyword 5, Keyword 6, Keyword 7, Keyword 8, Keyword 9
% \end{keywords}
\keywords{
  cultural heritage \and
  Tróia \and
  digital tools \and
  online visitors \and
  rare artifacts  \and
  Roman Empire 
}
%falo em museu virtual?