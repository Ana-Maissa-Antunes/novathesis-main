%!TEX root = ../template.tex
%%%%%%%%%%%%%%%%%%%%%%%%%%%%%%%%%%%%%%%%%%%%%%%%%%%%%%%%%%%%%%%%%%%%
%% abstract-pt.tex
%% NOVA thesis document file
%%
%% Abstract in Portuguese
%%%%%%%%%%%%%%%%%%%%%%%%%%%%%%%%%%%%%%%%%%%%%%%%%%%%%%%%%%%%%%%%%%%%

\typeout{NT FILE abstract-pt.tex}%

Nos dias de hoje, um tesouro histórico-cultural situado em Tróia permanece maioritariamente desconhecido.  
Este projeto estuda vidros luxuosos e únicos do período romano, encontrados em escavações realizadas em Tróia.


O objetivo desta dissertação é dar a conhecer esta valiosa descoberta através do desenvolvimento de diversas ferramentas que permitam a disseminação deste estudo, de modo a preservar o seu valor cultural e partilhar os resultados da sua investigação. 
Este projeto desafiante está atualmente em fase de arranque e visa a continuidade futura do seu desenvolvimento, com grande potencial para novos progressos.
Além disso, abrange várias áreas de Engenharia Informática, algumas das quais não abordadas em aula, constituindo assim uma excelente oportunidade para expandir o meu conhecimento técnico. 
Adicionalmente, ajudará a desenvolver algumas soft skills, como autonomia, criatividade, e pensamento crítico através do planeamento e estruturação do projeto, bem como a integração dos seus componentes. 


Durante o desenvolvimento desta proposta, serão implementadas plataformas interativas, incluindo um mapa interativo da Península Troiana, um repositório contendo dados sobre as escavações realizadas em Tróia e uma experiência de realidade aumentada (RA), que permitirá aos visitantes interagir com um objeto físico existente. 
Por fim, estes elementos serão integrados num ambiente digital sob a forma de modelos 3D.
Estas tecnologias irão enriquecer a experiência do público online das ruínas de Tróia, proporcionando-lhes a oportunidade de interagir com alguns dos artefactos mais raros do Império Romano encontrados numa campa luxuosa. 
Com a finalidade de impactar positivamente as pessoas, expandindo o seu conhecimento acerca deste importante sítio arqueológico e preservando o seu legado cultural.


% Palavras-chave do resumo em Português
% \begin{keywords}
% Palavra-chave 1, Palavra-chave 2, Palavra-chave 3, Palavra-chave 4
% \end{keywords}
\keywords{
  tesouro histórico-cultural \and 
  Tróia \and
  tecnologias \and
  visitantes \and
  artefactos raros  \and
  Império Romano
  }
% to add an extra black line
