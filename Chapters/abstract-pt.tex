%!TEX root = ../template.tex
%%%%%%%%%%%%%%%%%%%%%%%%%%%%%%%%%%%%%%%%%%%%%%%%%%%%%%%%%%%%%%%%%%%%
%% abstract-pt.tex
%% NOVA thesis document file
%%
%% Abstract in Portuguese
%%%%%%%%%%%%%%%%%%%%%%%%%%%%%%%%%%%%%%%%%%%%%%%%%%%%%%%%%%%%%%%%%%%%

\typeout{NT FILE abstract-pt.tex}%


Nos dias de hoje, um tesouro histórico-cultural situado em Troia permanece maioritariamente desconhecido.  
Esta dissertação foca-se em artefactos de vidros luxuosos e únicos do período romano, encontrados recentemente numa tomba descoberta em Troia, Setúbal.

A necessidade de dar a conhecer esta valiosa descoberta levou à colaboração entre os Departamentos de Informática e Conservação e Restauro da NOVA FCT e arqueólogos das Ruínas Romanas de Troia. O projeto envolveu o desenvolvimento de um ambiente virtual para disseminar este estudo, preservar o seu valor cultural, partilhar os seus resultados, e potencializar o seu futuro progresso.

Este trabalho abrange a integração dos dados recolhidos, incluindo registos de escavações, intervenções e dados de artefactos do sítio arqueológico, e o desenvolvimento preliminar de uma aplicação de realidade virtual. Os dados recolhidos são armazenados numa base de dados que é um repositório central. O ambiente virtual proporciona uma imersão virtual numa visita a um túmulo inserido num recinto funerário, com dois objetos, e inclui um mapa base interativo da área ocidental do sítio arqueológico de Troia. O recinto funerário e os objetos foram integrados no ambiente virtual sob a forma de modelos 3D. Além disso, os utilizadores podem interagir com um painel virtual, visualizar detalhes sobre os objetos, alternar a planta do local e visualizar um perfil do funerário. Estes elementos permitem aos utilizadores explorar o recinto funerário e interagir com alguns dos artefactos mais raros do Império Romano.

Numa etapa final, a experiência foi testada com 28 participantes de diferentes contextos. Os resultados foram avaliados através de um questionário contendo a usabilidade da tarefa, o sentimento de presença e a experiência global do utilizador. A avaliação indicou um resultado global positivo, com algumas melhorias posteriormente implementadas com base no feedback dos utilizadores.


\keywords{
  Tesouro Histórico-Cultural \and 
  Troia \and
  Realidade Virtual \and
  Artefactos Raros  \and
  Império Romano
  }
% to add an extra black line
