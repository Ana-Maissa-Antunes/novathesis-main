%!TEX root = ../template.tex
%%%%%%%%%%%%%%%%%%%%%%%%%%%%%%%%%%%%%%%%%%%%%%%%%%%%%%%%%%%%%%%%%%%%
%% abstract-pt.tex
%% NOVA thesis document file
%%
%% Abstract in Portuguese
%%%%%%%%%%%%%%%%%%%%%%%%%%%%%%%%%%%%%%%%%%%%%%%%%%%%%%%%%%%%%%%%%%%%

\typeout{NT FILE abstract-pt.tex}%

Nos dias de hoje, um tesouro histórico-cultural situado em Troia permanece maioritariamente desconhecido.  
Este projeto estuda vidros luxuosos e únicos do período romano, encontrados recentemente numa luxuosa tomba descoberta em Troia, Setúbal.


A necessidade de dar a conhecer esta valiosa descoberta levou à colaboração entre os departamentos de Informática e de Conservação e Restauro da NOVA FCT e a contribuição de arqueólogos do sítio arqueológico de Tróia. O projeto envolverá o desenvolvimento de diversas ferramentas digitais que permitam a disseminação deste estudo, de modo a preservar o seu valor cultural e partilhar os resultados da sua investigação
e garantir a continuidade futura do seu desenvolvimento, com grande potencial para novos progressos.
% Além disso, abrange várias áreas de Engenharia Informática, algumas das quais não abordadas em aula, constituindo assim uma excelente oportunidade para expandir o meu conhecimento técnico. 
% Adicionalmente, ajudará a desenvolver algumas soft skills, como autonomia, criatividade, e pensamento crítico através do planeamento e estruturação do projeto, bem como a integração dos seus componentes. 

Esta dissertação abrange a análise e integração dos dados e artefactos recolhidos no local e o desenvolvimento preliminar de uma plataforma digital interativa que, no futuro, permitirá a divulgação destas descobertas ao público.
%Durante o desenvolvimento desta proposta, 
Esta plataforma deverá, no seu estado final, incluir um repositório contendo dados sobre as escavações realizadas em Troia e uma experiência de Realidade Virtual (RV), que permitirá uma imersão virtual a objetos físicos existentes, e um mapa interativo deste sítio arqueológico de Tróia. 
Por fim, estes elementos serão integrados num ambiente digital sob a forma de modelos 3D.
Estas tecnologias irão enriquecer a experiência dos visitantes das ruínas de Troia, proporcionando-lhes a oportunidade de interagir com alguns dos artefactos mais raros do Império Romano. 
Além disso, a plataforma resultante deverá impactar positivamente as pessoas, expandindo o seu conhecimento acerca deste importante sítio arqueológico e preservando o seu legado cultural.


% Palavras-chave do resumo em Português
% \begin{keywords}
% Palavra-chave 1, Palavra-chave 2, Palavra-chave 3, Palavra-chave 4
% \end{keywords}
\keywords{
  Tesouro Histórico-Cultural \and 
  Troia \and
  Ferramentas Digitais \and
  Realidade Virtual \and
  Artefactos Raros  \and
  Império Romano
  }
% to add an extra black line
